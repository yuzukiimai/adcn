\documentclass{jsarticle}
\usepackage[dvipdfmx]{graphicx}
\usepackage{subcaption}
\captionsetup[figure]{justification=centering}
\captionsetup[table]{justification=centering}
\usepackage[ipa]{pxchfon}

\begin{document}
\title{{\vspace*{-30mm}}{\LARGE 0601特別講義レポート}}
\author{\large 千葉工業大学 先進工学部 未来ロボティクス学科 \vspace*{4mm}\\20C1015 今井悠月}
\date{}
\maketitle\vspace*{10mm}

\section*{問1}
人, 環境, ロボティクスの例を挙げてください.

\subsection*{回答}
私なりに「人, 環境, ロボティクス」を考えてみる. \hspace*{1zw}「人, 環境, ロボティクス」とは, この3要素が互いに影響を
及ぼし合うような関係であると考える. \hspace*{1zw}そのように考えると, 私達の身の回りにはそのような状況が多く存在する.
\hspace*{1zw}ここで具体例を挙げる.\hspace*{1zw}私は現在, ラーメン屋でアルバイトをしているが, 3ヶ月ほど前から
配膳ロボットが導入された.\hspace*{1zw}この配膳ロボットは, 出来上がった商品を人間が指定したテーブルへ自律移動し, 運ぶ
というものである.\hspace*{1zw}配膳後は, お客様にボタンを押してもらい, 自律移動で元の位置に戻ってくるといったシステムになっている.
\hspace*{1zw}しかし, ボタンを押してくれないお客様もわずかながら存在する.\hspace*{1zw}そのような場合, ロボットは自律的に
元の位置には戻ろうとせず, 現在地に留まることになる.\hspace*{1zw}また, 自律的に移動する故に興味を抱き, 近づいてくることで, 
ロボットの通り道が塞がれてしまう場面も存在する.\hspace*{1zw}このようなケースは特に小さな子供が店内にいる際に頻発する.
さらに, 店内の様子は常に一定に保たれておらず, ロボットが認知できないような小さなゴミやガラスの破片が落ちていることがある.
\hspace*{1zw}そのような場合, 車輪駆動のロボットにおいては, 自律移動を妨げることになる.\\
\hspace*{1zw}上記のような問題は, ロボットと人, 環境が密接に関係しているからこそ起こるのである.

\vspace*{10mm}

\section*{問2}
SEAと油圧のメリット, デメリットを議論してください.

\subsection*{回答}
SEA(Series Elastic Actuator)のメリットとしては, 耐衝撃性, 安全性, エネルギーの保存や力の制御が挙げられる.
\hspace*{1zw}バネを用いているため, 柔軟性があり, 人的危害を加えにくい. \hspace*{1zw}また, 自身のアクチュエータ
を壊しにくいというのもメリットである.
\hspace*{1zw}デメリットとしては, バネを使用するため, 大きな力を発生させることが困難であると考えられる.\\
\hspace*{1zw}油圧の一番のメリットとしては, 大きな力を発生させることである.\hspace*{1zw}また, 剛性があり, 負荷の保持に関しても優れている.
\hspace*{1zw}デメリットととしては, やはり油を使うが故の危険性が挙げられる.\hspace*{1zw}油が漏れることで, 火災につながる恐れがある.
\hspace*{1zw}大きな力を出せる代わりに騒音が大きいというのもデメリットであろう.\hspace*{1zw}\\また, 油によってサビが発生することも考えられるため, 
メンテナンス性が悪いのではないだろうか.\\
\hspace*{1zw}SEAと油圧のどちらもメリットとデメリットが存在し, また, 相互で欠点を補うような関係になっているような印象である.
\hspace*{1zw}よって, 求められる作業に応じて適している機構を選択するのが望ましいだろう.\\\vspace*{4mm}
\section*{感想}
この度は, 貴重なお話をしてくださりありがとうございました. \hspace*{1zw}現場で開発する方々の声を聞ける機会というのは滅多にないため, 
非常に勉強になりました. \hspace*{1zw}私は現在, 林原教授のご指導の下, 車輪駆動型の自律移動ロボットの研究を行っています.
研究ではROSを使用しているため, 海洋のロボットのお話はとても興味を惹く内容でした. 3DLiDARやIMU, は私達も使用しているセンサであるため, 親近感が湧きました.\\
\hspace*{1zw}そこで海洋ロボットについて, 一つ質問なのですが, レーダというのはどのような仕組みになっているのでしょうか?\\
\hspace*{1zw}回転するようですが, 割と上部に搭載されているため, 主にどこを見て, なにをどのように測定しているのか疑問に思いました.

\end{document}